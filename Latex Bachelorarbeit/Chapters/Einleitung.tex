\chapter{Einleitung}
\label{cha:Einleitung}

In den letzten Jahrzehnten ist der Energieverbrauch und die damit einhergehenden Folgen für die Umwelt zu einem viel diskutierten Thema geworden. Im Fokus dieser Diskussion liegt allerdings nicht nur der Verbrauch der Energie sondern auch die Steigerung der Energieeffizienz, durch die eine Reduktion des Energiebedarfs umgesetzt werden kann.\\
Fast 40 \% des Primärenergieverbrauchs in Deutschland sind dem Gebäudesektor zuzuordnen. Hierbei ist nicht nur der prozentual größere Wohngebäudeanteil von Bedeutung, im Hinblick auf die Einsparung von Primärenergie spielen auch Nichtwohngebäude eine wesentliche Rolle. Nach Betrachtung der Anzahl von Wohngebäuden (18,2 Millionen) und Nichtwohngebäuden (1,7 Millionen) und dem jeweiligen Anteil des Energieverbrauchs von 65\% bei Wohn- und 35\% bei Nichtwohngebäuden wird deutlich, dass bei Nichtwohngebäuden ein großes Optimierungspotential vorhanden ist. Durch die Forderungen der Politik durch die Einführung der Novellen der Energieeinsparverordnung (EnEV 2014, EnEv 2016) soll die Effizienz des Gebäudesektors optimiert werden. Das ausgesprochene Ziel ist ein nahezu klimaneutraler Gebäudebestand im Jahr 2050. Folglich sind Auflagen für Neubauten, aber auch für bestehende Gebäude umzusetzen. \\
\\
Im Rahmen des Projekts En-Eff-Campus: Roadmap der RWTH Aachen, welches der Reduktion von Primärenergie der Liegenschaften der RWTH dienen soll, wird der Gebäudebestand der Universität systematisch aufgenommen und bezüglich seiner Optimierungspotentiale untersucht. Zielsetzung ist die Reduzierung der Primärenergie um 50\% bis zum Jahr 2025 bezogen auf die im Jahr 2013/2014 aufgenommenen Daten. Mithilfe der Datengrundlage des IST-Bestandes sowie dynamischen Quartiersimulationen sollen sowohl die ökonomischsten als auch die effizientesten Optimierungsmaßnahmen herausgearbeitet werden. \\
\\
Auf Basis der erfassten Daten wird im Zuge dieser Bachelorarbeit der Ist-Zustand eines für die Campusse der RWTH repräsentativen Gebäudes abgebildet. Durch die Betrachtung eines Leuchtturm-Gebäudes soll nach Möglichkeit ein Optimierungspotential herausgearbeitet werden, das auf mehrere Gebäude der Liegenschaften der RWTH anwendbar ist. Hierbei liegt der Fokus auf der Optimierung der Anlagentechnik der Gebäude. \\

Zunächst wird eine Analyse der vorliegenden Daten des Gebäudebestands der Universität durchgeführt. Anhand der Festlegung verschiedener Kriterien wird dieser eingegrenzt und es folgt die Auswahl eines repräsentativen Gebäudes. \\
Im Folgenden wird das Gebäude im Rahmen von Gebäudebegehungen und der Sichtung von vorliegenden Daten genauer betrachtet und beschrieben. Hierbei wird vermehrt auf die Anlagentechnik des Gebäudes eingegangen.\\
Daraufhin wird das Gebäude im Simulationsprogramm Dymola abgebildet. Es folgt eine thermische Verbrauchs-Bedarfsanalyse. Im Rahmen dieser Analyse wird auf getroffene Annahmen und eventuelle Differenzen zwischen dem simulierten Bedarf und dem vorliegenden Verbrauch eingegangen. \\
Als Hauptteil dieser Arbeit werden verschiedene Optimierungspotentiale, hauptsächlich für eine höhere Effizienz der Anlagentechnik, herausgearbeitet und bezüglich ihrer Reduktion der Heizenergie bewertet. 
Abschließend folgt die Zusammenfassung der ermittelten Ergebnisse und ein Ausblick für das Projekt der RWTH. 


