\chapter{Beschreibung des Gebäudemodells}
\label{cha:Beschreibung des Gebäudemodells}

Die Abbildung sowie die Simulation der vorgestellten Objekts erfolgt mit dem Simulationsprogramm Dymola. Das für die Simulation verwendete Modell trägt den Namen „Multizone Equipped“ und entstammt der institutseigenen Modelica-Bibliothek AixLib. Es wurde im Rahmen des EnEff:CampusRoadmap entwickelt. Das Modell stellt ein Nichtwohngebäude dar, welches auf einer thermischen Zonierung basiert. Komponenten der Anlagentechnik wie Heizung und  Kühlung sind im Modell integriert, ebenfalls wie ein Raumlufttechnische Anlage. Die Eigenschaften der Gebäudehülle fließen ebenfalls in das Modell mit ein. Es existieren bereits vorgefertigte Zonen, deren Parameter lediglich angepasst werden müssen. Aufgrund der Komplexität der Erfassung eines bestehenden Gebäudes und der meist spärlichen Datenlage, kommt es in einigen Bereichen zur Vereinfachungen der Abbildung.\\
Das Gebäude wird beispielsweise nicht Raum für Raum in das Modell eingefügt, sondern die Abbildung wird durch die Einteilung der Grundfläche in verschiedene Bereiche unterteilt. Diese Bereiche entsprechen den vordefinierten Zonen des Modells. Diese wurden in Anlehnung an DIN V 18599 und Merkblatt SIA 2024 erstellt. Die Gebäudephysik wird mithilfe einer Ersatzinnenwand sowie mit einer Ersatzaußenwand dargestellt. Diese werden in Anlehnung an die Geometrie der einzelnen Zonen erstellt. Das thermische Verhalten der realen Bauteile fließt somit durch die Ersatzwände in das Modell ein. Die Abbildung der Gebäudephysik erfolgt über standardisierte Werte aus einer Datenbank, die durch das Baujahr charakterisiert werden. So werden repräsentative Werte für den Wärmedurchgangskoeffizient der Wände und der dadurch resultierende Wärmeverlust nach Außen ermittelt.\\
Eine weitere Vereinfachung wird im Bereich der Fenster angenommen. Aufgrund der nicht Grundriss getreuen Aufteilung der Räume, wird die Sonneneinstrahlung auf Fenster und Außenwandflächen mithilfe von Gewichtungsfaktoren ermittelt, die auf dem relativen Anteil der Flächen basieren. Die vorhandene Fensterfläche wird also auch aufgrund von Erfahrungswerten abgeschätzt.\\
Der Einfluss von Gebäudenutzern und Außenfaktoren wie Wetter können separat an das Modell angepasst werden. Die Nutzung durch Personen fließt als Wärmeabgabe von Personen, Maschinen und der Beleuchtung in das Modell mit ein. Durch das Einbinden von Wetterdaten können lokale Sonneneinstrahlung sowie der Außentemperatureinfluss an verschiedene Orte angepasst werden. \\
(Fuchs,Marcus)\\
\\
Zunächst werden die Gebäude im Hinblick auf die Existenz verschiedener Zonen untersucht. Die im betrachteten Gebäude vorhandenen Zonen Sanitary, Storage, Laboratory, Office, Meeting und Lecture bzw. Cooling Zone werden mit der von ihren eingenommenen Fläche nach der obigen Beschreibung in das Gebäudemodell eingefügt. Parameter wie die Intensität der Beleuchtung sowie die Anzahl an Personen und Maschinen sind ebenfalls von der Nutzungsart der Zone abhängig. \\
Da beide Gebäude demselben Institut angehören und laut dem Bauwerkzuordnungskatalog derselben Kategorie entsprechen, sind die Nutzungsbereiche ähnlich aufgebaut. Die vorhandenen Nutzungsarten, die in beiden Gebäuden vorkommen, sind Labore, Büros,Verkehrsflächen, Meetings, Lager und sanitäre Bereiche. Zusätzlich zu diesen Nutzungsarten enthält Gebäude 2030 einen Hörsaal und eine Bibliothek, Gebäude 2031 verfügt über eine gekühlte Laborzone.\\
\\
Im Rahmen der Zonierung ist die Flächegröße der einzelnen Zonen von großer Bedeutung. Im Fall des Hauptgebäudes werden die Daten mithilfe der Grundrisses und der Angabe der Nettogrundfläche ermittelt. Anfragen nach genaueren Daten beim BLB NRW blieben ohne Ergebnis. Für den Erweiterungsbau 2031 liegt ein Grundriss mit Flächenangaben vor, auf dessen Basis die Zonierung erstellt wurde.\\
Um eine möglichste genaue Abbildung der Gebäude zu gewährleisten, sind mehrfach Ortsbegehungen durchgeführt worden. Im Zuge dieser Begehungen wurden die relevante Anlagentechnik für das thermische Energiesystem unter die Lupe genommen. Somit wurden jeweils des Heizungskeller und die Technikräume beider Gebäude besichtigt. \\
In einer weiteren Begehungen sind die Raumtemperaturen der einzelnen Zonen gemessen worden. Da die Messung der Temperaturen lediglich an zwei Tagen im März diesen Jahres vorgenommen wurden, sind die Werte nicht repräsentativ. Für eine repräsentatives Ergebnis der Raumtemperaturen wäre die Messung über einen längeren Zeitraum erforderlich gewesen. Dies ist aufgrund des zeitlichen Rahmens und des großen Aufwands nicht zu leisten. Bei der Messung steht viel mehr eine realistische Einstellung der Raumtemperaturen für die Simulation im Fokus. Die Messungen verdeutlichen zudem den großen Einfluss der nutzenden Personen. \\
Diese Ortsbegehung war zudem sinnvoll, da ein Kontakt zu den Nutzern des Gebäudes hergestellt werden konnte. Vermehrt wurde über die Funktionalität der Heizung geklagt.\\

Alle Zonen sind thermisch konditioniert und werden über eine statische Heizung mit Wärme versorgt. Die Lüftungsanlage in beiden Gebäuden dient nur der Versorgung des Laborbereichs, des Hörsaals und der Bibliothek(LAUT UIWBERICHT AUCH DEN FLUR). Die Lüftungsanlage dient neben dem Austauschen der Raumluft lediglich der Erwärmung der Luft. Außer in der Bibliothek ist keine Kühlung der Luft vorgesehen. Die Bibliothek, die in Gebäude 2030 in Form eines Lesesaals in Erscheinung tritt, wird in diesem Modell nicht in die Betrachtung miteinbezogen. Zwar weist sie mit einem Aggregat für die Enfeuchtung der Luft eine andere Konditionierung als die restlichen Zonen auf, allerdings nimmt sie mit einem Flächenwert von ca. 36 m² lediglich einen prozentualen Wert von 0,46\% der Nettogesamtfläche (inklusive Kellergeschosse) ein. Damit liegt der Wert noch unter 1\%, sodass die Zone laut DIN V 18599 10 trotz der Unterscheidung in der Konditionierung einer anderen Zone zugeordnet werden kann.\\
Da die Laborzone die einzige Zone ist, in der eine Konditionierung der Luft durch eine Raumlufttechnische Anlage vorliegt, wird die Bibliothek diesem Bereich zugeteilt. \\
Die gekühlte Laborzone des Gebäudes 2031 umfasst eine Fläche von 150 m² und setzt sich aus fünf Laborräumen zusammen. Diese Zone wird nicht nur beheizt und belüftet, sondern auch mittels der Kälteanlage im Obergeschoss des Gebäudes gekühlt. Trotz der gleichen Nutzung wird die Zone gesondert betrachtet, da sich die Konditionierung von der anderen Laborzone unterscheidet.\\
\\
Zur Vereinfachung des Gebäudemodells werden die Untergeschosse beider Gebäude vernachlässigt. Diese Annahme kann getroffen werden, da die Untergeschosse nicht thermisch konditioniert sind. Die jeweiligen Nettogrundflächen sind für die Vereinfachung des Modells angepasst worden. Die Technikräume, die nicht dem Versorgungsbereich angehören, sind ebenfalls von der NGF der Gebäude subtrahiert worden. Somit wird lediglich die Energiebezugsfläche betrachtet. \\
\\
Die in das Modell eingebundenen Wetterdaten stammen aus dem Jahr 2013 und wurden von der Wetterstation in Melaten aufgezeichnet. Durch Daten aus dem Raum Aachen wird die Diskrepanz der Unterschiede auf Grundlage der Wetterdaten minimiert.\\
\\
Die Lüftungsanlage läuft laut DIN V 18599 Teil 10 in den Laboren 24 Stunden pro Tag. (Die Volumenströme der Anlage in beiden Gebäuden weichen stark von den vordefinierten Standardprofilen der Zone ab und wurden an das Gebäude angepasst.) Aufgrund der Nutzung als Chemiegebäude und dem Umgang mit Chemikalien, ist der Austausch der Laborluft von besonderer Bedeutung. Während der Betriebszeiten läuft die Lüftungsanlage mit ihrem maximalen Volumenstrom. Außerhalb der festgelegten Betriebszeit kann die Lüftungsanlage in jedem Labor einzeln weiterhin mit ihrem maximalen Volumenstrom betrieben werden. Dazu ist jedes Labor mit einem Schalter ausgestattet, welcher der Regelung der Anlage im Raum dient. Dadurch kann auch außerhalb der vordefinierten Arbeitszeiten das Labor genutzt werden. Außerhalb der Betriebszeiten wird die Anlage mit einem Mindestvolumenstrom betrieben, wie es in der DIN 18599 V 10 für Labore vorgeschrieben ist. Durch diese manuelle Einflussmöglichkeit wird das Abschätzen des Einflusses der Lüftungsanlage in der Simulation erschwert.  \\
Bei der Durchführung der Simulation wurde die Annahme getroffen, dass die Lüftungsanlage kontinuierlich mit dem maximalen Volumenstrom betrieben wird. Dies wird dadurch begründet, dass die Schalter aus Erfahrungsberichten einiger Nutzer regelmäßig nicht ausgeschaltet werden, sondern der manuelle Schalter sich dauerhaft in der Einstellung des vollen Betriebes befindet. \\
\\
Die gekühlte Zone resultiert aus der Zusammenfassung von fünf Räumen, die sich in der Realität auf 3 verschiedene Etagen aufgeteilt sind. Die Kühlung erfolgt in drei dieser Räumen über einen Umluftkühler, in den zwei verbleibenden Laboren sind Kühldecken installiert. Die Leistung umfasst insgesamt 17,8kW, wobei die Umluftkühler mit je 5kW und die Kühldecken 1,4kW pro Aggregat beanspruchen. Die Zusammenfassung der einzelnen Räume zu einer Zone erfolgt ebenfalls auf Grundlage der DIN V 18599. 
Zwar unterscheidet sich die Umsetzung der Konditionierung aufgrund der Nutzung von Umluftkühlern und Kühldecken, jedoch dienen beide Aggregate der Kühlung. Eine weitere Unterteilung in zwei verschiedene Kühlzonen aufgrund der unterschiedlichen Umsetzung der Konditionierung erscheint in Anbetracht der Flächengröße nicht als sinnvoll.\\
\\
Die erste Durchführung der Simulation dient eines Abgleichs zwischen dem Verbrauch von Heizenergie und dem berechneten Bedarf. Die Analyse erfolgt unter Berücksichtigung von verschiedenen Vereinfachungen. Zum einen wird das Gebäude nicht Zentimetergenau in das Simulationsmodell übertragen. Durch die Einteilung in Zonen und die vereinfachte Ermittlung der Gebäudephysik ist mit einer Abweichung zwischen Verbrauch und Bedarf zu rechnen. Zudem kommt der Nutzereinfluss hinzu, der im Modell bis zu einem bestimmten Grad zwar miteinbezogen werden kann, in der Realität allerdings schwer einzuschätzen ist. 
Aufgrund mangelnder Informationen bezüglich der Gebäudephysik sind Durchlässigkeit von Fenstern und Außenwänden nur bedingt abschätzbar. Die Dichtheit der Gebäudehülle kann daher nicht zuverlässig eingeschätzt werden.\\
Zudem ist die Warmwasserbereitung im Hauptgebäude, die ebenfalls durch Fernwärme versorgt wird, nicht im Gebäudemodell enthalten. \\