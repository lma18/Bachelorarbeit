\chapter{Beschreibung des Gebäudemodells}
\label{cha:Beschreibung des Gebäudemodells}

Die Abbildung sowie die Simulation der vorgestellten Objekts erfolgt mit dem Simulationsprogramm Dymola. Im Folgenden Kapitel wird das verwendete Gebäudemodell vorgestellt und beschrieben. \\
\\
Das für die Simulation verwendete Modell trägt den Namen „Multizone Equipped“ und entstammt der institutseigenen Modelica-Bibliothek AixLib. Es wurde im Rahmen des EnEff:CampusRoadmap entwickelt. Das Modell stellt ein Nichtwohngebäude dar, welches auf einer thermischen Zonierung basiert. Im Bereich der Nichtwohngebäude stehen jeweils noch verschiedene Gebäudetypen zur Auswahl. Aufgrund des sehr hohen Anteils an Laborfläche werden beide Gebäude dem Gebäudetyp Institut 8 zugeteilt. Komponenten der Anlagentechnik wie Heizung und  Kühlung sind im Modell integriert,eine Raumlufttechnische Anlage ist ebenfalls vorhanden. Die Eigenschaften der Gebäudehülle fließen in das Modell mit ein. Es existieren bereits vorgefertigte Zonen, deren Parameter lediglich angepasst werden müssen. Die Erstellung dieser Zonen erfolgt in Anlehnung an DIN V 18599 und Merkblatt SIA 2024. Die Eingabe erfolgt über ein Eingabewerkzeug mit Namen Teaser. Mithilfe dieses Werkzeuges werden die Größen der Ersatzwände abgeschätzt mit den zugehörigen Faktoren wie Wärmedurchgang der Wände. Parameter wie die Intensität der Beleuchtung sowie die Anzahl an Personen und Maschinen sind ebenfalls von der Nutzungsart der Zone abhängig und werden über Teaser an die jeweilige Zone angepasst. Das thermische Verhalten der realen Bauteile fließt somit durch die Ersatzwände in das Modell ein.  Aufgrund der Komplexität der Erfassung eines bestehenden Gebäudes und der meist spärlichen Datenlage, müssen teilweise Vereinfachungen in Kauf genommen werden.\\
\\
Das Gebäude wird nicht Raum für Raum in das Modell eingefügt, sondern die Abbildung wird durch die Einteilung der Grundfläche in verschiedene Bereiche unterteilt. Diese Bereiche entsprechen schlussendlich den vordefinierten Zonen des Modells. 
Die Eingabe mit Teaser erfolgt über den auf eine ganze Zahl gerundeten prozentualen Anteil an der im Modell betrachteten Gesamtfläche. 
Aufgrund der nicht maßstabsgetreuen Abbildung der Geometrie kann beispielsweise durch die äußeren Abmessungen ein abweichendes Oberflächen-Volumen-Verhältnis auftreten. Dies beeinflusst sowohl den Wärmeverlust über Flächen mit Kontakt zur Außentemperatur, als auch ein verändertes Einwirken von Strahlung.\\
Eine weitere Abweichung stellt die Zonierung dar. Bei der Errichtung neuer Gebäude wird Wert auf eine Zonierung gelegt, sodass Bereiche mit selber Nutzung und Konditionierung sich in unmittelbarer Nähe befinden. Dadurch wird beim Vorhandensein eine Kühlzone lediglich dieser abgesteckte Bereich gekühlt und nicht mehrere einzelne Räume.\\
Die Gebäudephysik wird mithilfe einer Ersatzinnenwand sowie mit einer Ersatzaußenwand dargestellt. Diese werden in Anlehnung an die Geometrie der einzelnen Zonen erstellt. Die Abbildung der Gebäudephysik erfolgt über standardisierte Werte aus einer Datenbank, die durch das Baujahr charakterisiert werden. So werden repräsentative Werte für den Wärmedurchgangskoeffizient der Wände und der dadurch resultierende Wärmeverlust nach Außen ermittelt.\\
Eine weitere Vereinfachung wird im Bereich der Fenster angenommen. Aufgrund der nicht Grundriss getreuen Aufteilung der Räume, wird die Sonneneinstrahlung auf Fenster und Außenwandflächen mithilfe von Gewichtungsfaktoren ermittelt, die auf dem relativen Anteil der Flächen basieren. Die vorhandene Fensterfläche wird also auch aufgrund von Erfahrungswerten abgeschätzt.\\
Der Einfluss von Gebäudenutzern und weiteren Faktoren wie das Wetter können separat an das Modell angepasst werden. Die Nutzung durch Personen fließt als Wärmeabgabe von Personen, Maschinen und der Beleuchtung in das Modell mit ein. Durch das Einbinden von Wetterdaten können lokale Sonneneinstrahlung sowie der Außentemperatureinfluss an verschiedene Orte angepasst werden. \\
(Fuchs,Marcus)\\
\\
\section{Vorbereitung der Simulation}
\label{sec:Vorbereitung der Simulation}

Im Rahmen dieses Unterkapitels wird der Aufbau des Gebäudes in der Simulation thematisiert. Desweiteren wird auf die Einstellung der Anlagentechnik eingegangen. \\
Zunächst werden die Gebäude im Hinblick auf die Existenz verschiedener Zonen untersucht. Im Rahmen der Zonierung ist die Flächengröße der einzelnen Zonen von großer Bedeutung. Mithilfe des Grundrisses konnten die unterschiedlichen Nutzungsarten der Räume ermittelt werden. Anfragen nach genauen Fächendaten beim BLB NRW blieben ohne Ergebnis. Da im Fall des Hauptgebäudes keine Flächenangaben im Grundriss vermerkt waren, werden die Daten mithilfe der Grundrisses und der Angabe der Nettogrundfläche ermittelt. Mithilfe eines an den Grundriss angepassten Maßstabs konnten die prozentualen Flächenanteile der einzelnen Zonen eingeschätzt werden. Aufgrund der vereinfachten Ermittlung mithilfe des Grundrisses muss von kleineren Abweichungen der ermittelten Flächendaten im Bezug auf die realen Daten ausgegangen werden. Durch den Bezug auf die Nettogrundfläche liegt schlussendlich die Grundfläche der einzelnen Zonen vor. Für den Erweiterungsbau 2031 liegt ein Grundriss mit Flächenangaben vor, auf dessen Basis die Zonierung erstellt wurde.\\
\\
Um eine möglichst genaue Abbildung der Gebäude zu gewährleisten, sind mehrfach Ortsbegehungen durchgeführt worden. Im Zuge dieser Begehungen wurden die für das thermische Energiesystem relevante Anlagentechnik unter die Lupe genommen. Somit wurden jeweils der Heizungskeller und die Technikräume beider Gebäude besichtigt. Wie bereits im Kapitel Anlagentechnik beschrieben wird die statische Heizung sowie die Lüftungsanlage mit Heizenergie versorgt. Die Berücksichtigung von der Warmwasserbereitung ist nur im Hauptgebäude erforderlich. Im Turm wird das Wasser dezentral mithilfe von Strom erwärmt.\\
Im Rahmen einer weiteren Begehungen sind die Raumtemperaturen der einzelnen Zonen gemessen worden. Da die Messung der Temperaturen lediglich an zwei Tagen im März diesen Jahres (2016) vorgenommen wurden, sind die Werte nicht repräsentativ. Für eine repräsentatives Ergebnis der Raumtemperaturen wäre die Messung über einen längeren Zeitraum erforderlich gewesen. Dies ist aufgrund des zeitlichen Rahmens und des großen Aufwands nicht zu leisten. Bei der Messung steht viel mehr eine grobe Einschätzung der Raumtemperaturen für die Simulation im Fokus. Die Messungen verdeutlichen zudem den großen Einfluss der nutzenden Personen. \\
Mit der Messung der Temperaturen einhergehend, wurde Kontakt zu den Nutzern des Gebäudes hergestellt. Diese zeigten sich interessiert und hatten ihrerseits Anmerkungen bezüglich der Effizienz der Heizungsanlage. Vermehrt wurde über die Funktionalität der Heizung, genauer über die fehlerhafte Regulierung, geklagt.\\
\\
Da beide Gebäude demselben Institut angehören und laut dem Bauwerkzuordnungskatalog derselben Kategorie entsprechen, sind die Nutzungsbereiche ähnlich aufgebaut. Die übereinstimmenden Nutzungsarten sind Labore, Büros, Verkehrsflächen, Meetings, Lager und sanitäre Bereiche. Zusätzlich zu diesen Nutzungsarten enthält Gebäude 2030 einen Hörsaal und eine Bibliothek, Gebäude 2031 verfügt über eine gekühlte Laborzone.\\
\\
Wie bereits im Teil der Grundlagen beschrieben wurde, erfolgt die Zonierung auf Basis der Nutzung und der vorhandenen Konditionierung der vorliegenden Bereiche. Aus den Nutzungsbereichen werden folgende Zonen gebildet: Sanitary, Storage, Laboratory, Office, Meeting und Lecture bzw. Cooling Zone werden mit der von ihren eingenommenen Fläche in das Gebäudemodell eingefügt. Die Größe der jeweiligen Zone wird in der folgenden Abbildung verdeutlicht. \\
\\
Alle Zonen sind thermisch konditioniert und werden über eine statische Heizung mit Wärme versorgt. Die Lüftungsanlage in beiden Gebäuden dient nur der Versorgung des Laborbereichs, des Hörsaals und der Bibliothek(LAUT UIWBERICHT AUCH DEN FLUR). Die Lüftungsanlage dient neben dem Austauschen der Raumluft lediglich der Erwärmung der Luft. \\
Außer in der Bibliothek ist keine Kühlung der Luft vorgesehen. Die Bibliothek, die in Form eines Lesesaals in Erscheinung tritt, wird in diesem Modell nicht als eigene Zone in die Betrachtung miteinbezogen. Zwar weist sie mit einem Aggregat für die Entfeuchtung der Luft eine andere Konditionierung als die restlichen Zonen auf, allerdings nimmt sie mit einem Flächenwert von ca. 36 m² lediglich einen prozentualen Wert von 0,46\% der Nettogesamtfläche (inklusive Kellergeschosse) ein. Damit liegt der Wert noch unter 1\%, sodass die Zone laut DIN V 18599 10 trotz der Unterscheidung in der Konditionierung einer anderen Zone zugeordnet werden kann. Im Folgenden wird die Bibliothek als Teil der Laborzone betrachtet. \\
\\
Die gekühlte Laborzone des Gebäudes 2031 umfasst eine Fläche von 150 m². Diese Zone wird nicht nur beheizt und belüftet, sondern auch mittels der Kälteanlage im Obergeschoss des Gebäudes gekühlt. Trotz der identischen Nutzung als Labor wird die Zone gesondert betrachtet, da sich die Konditionierung von der anderen Laborzone unterscheidet. \\
Die Zone resultiert aus der Zusammenfassung von fünf Räumen, die sich in der Realität auf 3 verschiedene Etagen aufgeteilt sind. Die Kühlung erfolgt in drei dieser Räumen über einen Umluftkühler, in den zwei verbleibenden Laboren sind Kühldecken installiert. Die Leistung umfasst insgesamt 17,8kW, wobei die Umluftkühler mit je 5kW und die Kühldecken 1,4kW pro Aggregat beanspruchen. Die Zusammenfassung der einzelnen Räume zu einer Zone erfolgt ebenfalls auf Grundlage der DIN V 18599. 
Zwar unterscheidet sich die Umsetzung der Konditionierung aufgrund der Nutzung von Umluftkühlern und Kühldecken, jedoch dienen beide Aggregate der Kühlung. Eine weitere Unterteilung in zwei verschiedene Kühlzonen aufgrund der unterschiedlichen Umsetzung der Konditionierung erscheint in Anbetracht der Flächengröße nicht als sinnvoll.\\
\\
Als weitere Vereinfachung des Gebäudemodells werden die Räume, die die Lüftungsanlage sowie die Heizung beherbergen nicht im Modell mit eingefügt. In diesen Bereichen des Gebäudes liegt keine thermische Konditionierung vor. Daher ist die Betrachtung im Modell nicht von Relevanz und wird bei beiden Gebäuden vernachlässigt. Der Versorgungsbereich der Heizung umfasst somit die gesamte in das Modell übertragene Fläche. Die jeweiligen Nettogrundflächen sind daher auf für die Vereinfachung des Modells angepasst worden. Die Technikräume, die nicht dem Versorgungsbereich angehören, sind ebenfalls von der NGF der Gebäude subtrahiert worden. Somit wird lediglich die Energiebezugsfläche betrachtet. \\
\\
Mit einer Geschosshöhe von 5,1 m weisen die Räumen einen großes Volumen auf. Zwar sind die Decken in manchen Teilen des Gebäudes abgehängt, die gesamte Laborzone sowie ein Großteil des Flurs sind allerdings nicht mit abgehängten Decken versehen. Für die Durchführung der Simulation ist eine Geschosshöhe von 5,1 m angenommen worden.\\
\\
Die in das Modell eingebundenen Wetterdaten stammen aus dem Jahr 2013 und wurden von der Wetterstation in Melaten aufgezeichnet. Durch Daten aus dem Raum Aachen wird die Diskrepanz der Unterschiede auf Grundlage der Wetterdaten minimiert.\\
\\
Die Lüftungsanlage läuft laut DIN V 18599 Teil 10 in den Laboren 24 Stunden pro Tag. (Die Volumenströme der Anlage in beiden Gebäuden weichen stark von den vordefinierten Standardprofilen der Zone ab und wurden an das Gebäude angepasst.) Aufgrund der Nutzung als Chemiegebäude und dem Umgang mit Chemikalien, ist der Austausch der Laborluft von besonderer Bedeutung. Während der Betriebszeiten läuft die Lüftungsanlage mit ihrem maximalen Volumenstrom. Außerhalb der festgelegten Betriebszeit kann die Lüftungsanlage in jedem Labor einzeln weiterhin mit ihrem maximalen Volumenstrom betrieben werden. Dazu ist jedes Labor mit einem Schalter ausgestattet, welcher der Regelung der Anlage im Raum dient. Durch diese Vorkehrung kann das Labor auch außerhalb der vordefinierten Arbeitszeiten genutzt werden. Außerhalb der Betriebszeiten wird die Anlage mit einem Mindestvolumenstrom betrieben, wie es in der DIN 18599 V 10 für Labore vorgeschrieben ist. Aufgrund der manuellen Einflussmöglichkeit wird das Abschätzen des Einflusses der Lüftungsanlage in der Simulation erschwert.\\
Bei der Durchführung der Simulation wurde die Annahme getroffen, dass die Lüftungsanlage kontinuierlich mit dem maximalen Volumenstrom betrieben wird. Dies wird dadurch begründet, dass die Schalter aus Erfahrungsberichten einiger Nutzer regelmäßig nicht ausgeschaltet werden, sondern der manuelle Schalter sich dauerhaft in der Einstellung des vollen Betriebes befindet. \\
In der Laborzone wird zudem ein Unterdruck erzeugt, um die Kontamination der benachbarten Räume durch stoffliche Lasten zu verhindern. Dies hat zur Folge, dass der abführende Volumenstrom mehr Luft aus dem Raum entnimmt. So kommt es bei einer fehlerhaften Abdichtung zwischen zwei Räumen zu einem Einströmen der Luft in das Labor anstatt andersherum.\\
\\
Die Temperatur der Zonen wird auf 22°C eingestellt. Dies stellt laut DIN 18599 V die empfohlene Solltemperatur der Heizung für Labore dar. Die Temperaturmessungen im Rahmen der Gebäudebegehung weisen mit einem Schnitt im Hauptgebäude von 21,6°C und im Turm von 22,2°C ähnliche Werte auf. Während die Festschreibung der Solltemperaturen der Heizung in den weiteren Zonen laut DIN 18599 bei 21°C liegt, schwankt dieser Wert bezogen auf die in der Praxis eigenen ermittelten, aber nicht repräsentativen Messungen. 
Im Hauptgebäude liegen verglichen mit der Solltemperatur geringfügig höhere Temperaturen vor. So wurden in den Zonen Storage und Meeting Werte von 21,5°C gemessen, in den Zonen Office und Sanitary lagen die auf die Fläche gemittelten Werte bei 21,4°C. Lediglich die Zonen Lecture, die ausschließich aus dem Hörsaal besteht, und die Zone Floor sind mit Werten von 20,6°C und 20,7°C unter dem Sollwert der DIN.
Die Temperaturverteilung im Turm zeigte eine ähnliche Temperaturverteilung. Die Zone Floor kennzeichnete ebenfalls eine relativ geringe Temperatur von 20,7°C. In den Zonen Office und Storage liegen die Temperaturen mit 22,3°C und 22°C mit 1 °C über dem Richtwert. Die Zone Meeting wies am Tag der Messung einen Wert von 21,2°C auf. 

Anhand der standardisierten Werte für die Bauteile, die durch Teaser in das Gebäudemodell mit einfließen, wird ein Wärmedurchgangskoeffizient der Fenster von 3W/m²K erreicht. Ein vergleichbarer Wert von 2,9w/m²K war das Ergebnis der Gruppe, welche die Gebäudephysik mithilfe der Software der Firma SolarComputer untersucht hat. 

Der Jahresheizwärmebedarf setzt sich aus Wärmesenken und Wärmequellen zusammen. Wärmesenken stellen hier die Transmissionswärme und die Lüftungswärme dar. Interne Wärmequellen und den Einfluss von solarer Strahlung wirkt den Energieverlusten durch die Wärmesenken entgegen. 
Durch den Jahresheizenergiebedarf wird neben dem Jahresheizwärmebedarf auch die Erwämung von Wasser und Qt und Qr berücksichtigt.
\section{Bedarfs-Verbrauchsanalyse}
\label{sec:Bedarfs-Verbrauchsanalyse}
Die Ergebnisse der Simulation und die Messdaten werden im Folgenden verglichen. Es wird der Heizenergieverbrauch pro Jahr betrachtet. Dieser Wert bezieht sich auf die Heizenergie, angegeben in kWh, die pro Jahr und Quadratmeter Nettogrundfläche verbraucht wird. Um eine Vergleichbarkeit des Kennwertes über mehrere Jahre zu schaffen, geht die Witterung in die Berechnung mit ein. Der Vergleich von gleichartigen Gebäuden an verschiedenen Standorten ist durch die Bereinigung der Witterung ebenfalls möglich. Aufgrund einer sehr milden Witterung im Jahr 2014 liegt der Heizenergieverbrauchskennwerte um 14\% niedriger als im Jahr 2013. Die große Differenz der Daten für das Jahr 2013 und 2014 im Energiebericht lässt sich zum Teil durch die klimatischen Verhältnisse erklären. \\
\\
Die erste Durchführung der Simulation dient eines Abgleichs zwischen dem Verbrauch von Heizenergie und dem berechneten Bedarf. Das Ergebnis der Simulation verdeutlicht den Bedarf an Energie, der anhand von technischen Daten der Anlage und Daten über die Bauphysik des Gebäudes berechnet werden kann. Der Verbrauch bezieht die nutzerspezifische Komponente und weitere Einflussfaktoren wie das Klima und das reale Verhalten der Regelungs- und Anlagentechnik mit ein. Die Simulationszeit beträgt 1 Jahr. \\
Die Analyse erfolgt unter Berücksichtigung der verschiedenen Vereinfachungen, die im vorherigen Kapitel bereits angesprochen wurden. Zum einen wird das Gebäude nicht Zentimetergenau in das Simulationsmodell übertragen. Durch die Einteilung in Zonen und die vereinfachte Ermittlung der Gebäudephysik ist mit einer Abweichung zwischen Verbrauch und Bedarf zu rechnen. Zudem kommt der Nutzereinfluss hinzu, der im Modell bis zu einem bestimmten Grad zwar miteinbezogen werden kann, in der Realität allerdings schwer einzuschätzen ist.
Aufgrund mangelnder Informationen bezüglich der Gebäudephysik ist der über die Bauteile zustande kommende Wärmeverlust nur bedingt abschätzbar. Die Dichtheit der Gebäudehülle kann ebenfalls nicht zuverlässig eingeschätzt werden.\\

Zudem ist die Warmwasserbereitung im Hauptgebäude, die ebenfalls durch Fernwärme versorgt wird, nicht im Gebäudemodell enthalten. Dadurch wird eine Korrektur des Heizenergieverbrauchswerts notwendig. Die Einbeziehung der Warmwasserbereitung erfolgt anhand der Technischen Daten des verwendeten Warmwasserspeichers. Der Speicher mit einem Volumen von 200 l benötigt zu Aufrechterhaltung der Temperatur laut Hersteller 1,46kWh/d. Dies entspricht einem Wert von 532,9 kWh/a.\\
Verwendet man den Richtwert der DIN 18599 (3Wh/m²d *365) für den Warmwasserbedarf eines Laborgebäudes ergibt sich ein Heizenergiebedarf von 10,95 kWh/m²a, der zusätzlich zu dem in der Simulation ermittelten Wert anfällt. Die Energie, die für das Halten der Temperatur im Speicher aufgewendet wird, macht mit einem Wert von 0,15kWh/m²a nur einen geringen Anteil bezogen auf den Heizenergiebedarf aus.