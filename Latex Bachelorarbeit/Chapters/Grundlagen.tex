\chapter{Rechtliche Grundlagen}
\label{cha:Rechtliche Grundlagen}

Die Notwendigkeit des Handlungsbedarfs im Bezug auf die Einsparung von Energie im Bauwesen ist schon seit Jahrzehnten Thema in der Politik. Ein Resultat ist das Energieeinsparungsgesetz(EnEG) aus dem Jahr 1976, unmittelbar nach der ersten Ölkrise, das in den Folgejahren (1980,2005,2009 und 2013) weiter verschärft und spezifiziert wurde. Die Novellierungen basieren auf der EU-Richtlinie 2009/91/EG zur Gesamtenergieeffizienz von Gebäuden. Das Gesetz dient des Einbaus von Wärmeschutz bei neu zu errichtenden Gebäuden sowie eine energiesparende Anlagentechnik, um vermeidbaren Energieverlusten vorzubeugen. Bei bestehenden Gebäuden werden bei Einbau oder Umbau der Anlagentechnik ebenfalls Anforderungen an die Energieeffizienz gestellt. [EnEG, 2013]  \\
Aus dem Energieeinsparungsgesetz geht die Wärmeschutzverordnung hervor, die gemeinsam mit der Verordnung für Heizungsanlagen die  Grundlage für die Energieeinsparverordnung (EnEV) bildet. Die EnEV ist im Jahr 2002 das erste Mal in Kraft getreten und  2004 aufgrund von Änderungen in den vorliegenden DIN Vorschriften novelliert worden. Im Jahr 2007 wurde eine Neufassung erstellt, die eine Trennung zwischen Wohngebäuden und Nichtwohngebäude beinhaltet. Seit 2007 wurden die Anforderungen der Verordnung im Jahr 2009 verschärft, eine weiter Novellierung erfolgte auf Basis der EnEG 2013 und hatte die EnEV 2014 zur Folge. Durch diese Neuerung ist die EU-Richtlinie 2010/31/EU in die nationale Gesetzgebung integriert worden. Die Richtlinie verfolgt eine europaweite Umsetzung von Mindeststandards für das gesamte Gebäude und schreibt zudem Energieausweise für Gebäude sowie regelmäßige Inspektionen der Anlagentechnik vor.\\
Zweck der EnEV ist die Einsparung von Energie im Gebäudesektor sowie die Umsetzung von energiesparenden Maßnahmen. Die Umsetzung erfolgt lediglich im Bereich der Wirtschaftlichkeit. Die Verordnung dient dem energiepolitischen Ziel im Jahr 2050 einen nahezu klimaneutralen Gebäudebestand zu erhalten.[EnEV,2014](vgl. enevonline) \\
\\
2013 wird im §2a des EnEG der Niedrigstenergie-Standard eingeführt, welcher vorschreibt, dass jedes Gebäude, welches nach 2020 gebaut wird und beheizt oder gekühlt werden muss, ein Niedrigstenergiegebäude sein muss. Für Nichtwohngebäude im Besitz der Behörden trifft diese Anforderung schon ab 01.01.2019 zu. Definiert wird ein Niedrigstenergiegebäude als ein Gebäude mit einer sehr guten Gesamtenergieeffizienz. Laut Definition ist der Energiebedarf des Gebäudes fast null bis sehr gering und die verwendete Energie wird zu einem wesentlichen Teil von erneuerbaren Quellen bereitgestellt.[EURichtlinie, Artikel 9] Die Einführung des Standards bei Neubauten ist im Rahmen der EnEV 2017 geplant. \\
\\
Der Anwendungsbereich der EnEV gilt für mit Energie beheizte oder gekühlte Gebäude, sowie Gebäude mit Heizungs-, Kühlungs und Beleuchtungstechnik und Raumlufttechnischen Anlagen. Denkmalgeschützte Gebäude und Gebäude, die besondere Anforderungen aufgrund ihrer Nutzung unterliegen, sind von der Verordnung ausgenommen. [EnEV 2014, §1]\\
Unterschieden wird im Wesentlichen zwischen Wohngebäuden und Nichtwohngebäuden. Im Folgenden wird lediglich auf die Nichtwohngebäude eingegangen, da dies der für diese Bachelorarbeit relevanten Kategorie entspricht.\\
Die Anforderungen für Nichtwohngebäude(§4) beziehen sich jeweils auf ein Referenzgebäude, welches in Nutzung, Ausrichtung, Nettogrundfläche sowie Geometrie mit dem betrachteten Gebäude übereinstimmt. Der Jahres-Primärenergiebedarf für die Anlagentechnik darf den des Referenzgebäudes nicht überschreiten. Die Berechnung des jeweiligen Verbrauchs wird durch eines der im Gesetzestext festgelegten Verfahren durchgeführt. Außerdem werden Grenzwerte für Wärmedurchgangskoeffizienten eingeführt, die sich auf die Umfassungsfläche des Gebäudes beziehen. Die in der Verordnung geforderten Verfahren zur Berechnung werden in der Vornormenreihe DIN V 18599 beschrieben, auf die im Folgenden genauer eingegangen wird.  \\
\\
Zur Förderung des Einsatzes Erneuerbarer Energien wird der durch diese gewonnene Strom vom Jahres-Primärenergieverbrauch abgezogen. Im Fall der Änderung, Erweiterung oder des Ausbaus von Gebäuden ist  eine Überschreitung der Grenzwerte von mehr als 40\% unzulässig. Die Aufrechterhaltung der energetischen Qualität muss bei jeder Änderungsmaßnahme, die mehr als 10\% des Gebäudes betrifft, gewährleistet sein.\\
Zur Effizienzsteigerung bei raumlufttechnischen Anlagen sieht die Verordnung eine selbsttätige Regelung von Volumenströmen zum einen in Abhängigkeit der Zeit oder zum anderen abhängig von stofflichen und thermischen Lasten ab einer Größe von 9m3/h /m2NGF vor. Bei einem Neubau von RLT-Anlagen ist der Einbau einer Wärmerückgewinnung verpflichtend. \\
\\ 
Die Einführung der EnEV beinhaltet zudem die Ausstellungspflicht von Energieausweisen. Unterschieden wird zwischen dem  Energiebedarfsausweis sowie dem Energieverbrauchsausweis. Bei Nichtwohngebäuden ist ein Energieverbrauchsausweis separat für Heizung und Strom auszustellen. Der Energieverbrauchsausweis gibt Auskunft über den Endenergieverbrauch für Heizung, Warmwasseraufbereitung, Kühlung, Lüftung und eingebaute Beleuchtung in kWh/m2 Nettogrundfläche. Der Energieverbrauch der Heizung wird mithilfe von Angaben aus der EnEV bezüglich der Variabilität der Witterung bereinigt. Die Ermittlung des Primärenergieverbrauch erfolgt auf Basis des Endenergieverbrauchs und unter Einbeziehung von Primärenergiefaktoren. Laut EnEV sollen die Daten des Endenergieverbrauchs  aus Abrechnungen von Heizkosten und andere Verbrauchsdaten aus Rechnungen stammen, die  einen zusammenhängenden Zeitraum von 36 Monaten umfassen müssen. Das Resultat stellt der durchschnittliche Energieverbrauch dieses Zeitraums dar.[EnEv,§18und 19] \\
Um festgeschriebene Ziel der Modernisierungsoffensive zu erfüllen, ist der Aussteller des Energieausweises ist laut §20 der EnEV zur Angabe von Modernisierungsempfehlungen verpflichtet, um über weitere Optimierungspotentiale bezüglich der Energieeffizienz aufzuklären.\\
\\
Die genaue Vorgehensweise und die Berechnungsverfahren der verschiedenen Komponenten der Anlagentechnik werden in der DIN V 18599 zusammengefasst. Die Vornormenreihe mit dem Titel "Energetische Bewertung von Gebäuden – Berechnung des Nutz-, End- und Primärenergiebedarfs für Heizung, Kühlung, Lüftung, Trinkwarmwasser und Beleuchtung" umfasst insgesamt 11 Teile. Sie dient der Ermittlung der Gesamtenergieeffizienz von Gebäuden unter Einbeziehung der Energie, die für die Versorgung der Bereiche Heizung, Kühlung, Lüftung, Trinkwarmwasser und Beleuchtung aufgewendet werden müssen. Die DIN V 18599 erfolgt einen integralen Ansatz, welcher die Berücksichtigung von Wechselwirkungen der verschiedenen Anlagen berücksichtigt. Auch in der DIN wird zwischen Wohn- und Nichtwohngebäuden unterschieden, ebenso wie zwischen Neu- und Bestandsbauten. Im Jahr 2005 wurde sie veröffentlicht und stellt somit ein weiteres Verfahren zur zuvor geltenden DIN 4701 aus dem Jahr 2002 dar. In der DIN V 4701 wird nicht zwischen Wohngebäuden und Nichtwohngebäuden unterschieden. Die EnEV bezieht sich weiterhin auf die DIN 4108, die Anforderungen an den Wärmeschutz stellt. In der EnEV 2014 besteht die Möglichkeit zwischen der Verwendung der Berechnungsverfahren der verschiedenen DINs zu wählen. \\
Durch die Berechnungsverfahren werden alle Energiemengen ermittelt, die die Bereiche Beheizung, Kühlung, Warmwasserbereitung, Beleuchtung und Raumlufttechnischer Anlagen betreffen. Relevanz für diese Bachelorarbeit haben insbesondere Teil 1 der Vornormenreihe sowie Teil 10, die im Folgenden detaillierter vorgestellt werden. \\
\\
Im ersten Teil sind die Definitionen und die zentralen Bilanzgleichungen für alle Normenteile festgelegt. Zudem wird die Vorgehensweise der energetischen Bilanzierung beschrieben. Zuvor erfolgt bei vorhandener Notwendigkeit eine Einteilung in verschiedenen Zonen. Dieser Fall liegt vor, wenn mehrere verschiedene Bereiche innerhalb des Gebäudes vorliegen, die sich in Bezug auf ihre Nutzung oder bezüglich ihrer Konditionierung unterscheiden.\\
Eine Zone ist laut DIN eine "grundlegende räumliche Berechnungseinheit für die Energiebilanzierung"(3.1.12). Die Erstellung der Zonierung kann in drei Schritten zusammengefasst werden. An erster Stelle werden die Bereiche in eine Zone eingeteilt, die die gleiche Nutzungsart aufweisen. Verschiedene Nutzungsprofile werden in Teil 10 der DIN V 18599 definiert und im weiteren Verlauf dieser Arbeit vorgestellt. Eine zweite Unterteilung erfolgt aufgrund der Art der Konditionierung. Die Konditionierung beschreibt das Vorhandensein bestimmter Anforderungen einer Gebäudezone in Bezug auf Raumklima, Beleuchtung oder Belüftung. Im Fall von unterschiedlicher Versorgungssysteme innerhalb einer Nutzungsart, die mit einer unterschiedlichen Luftkonditionierung einhergeht, ist der betrachtete Bereich in zwei Zonen zu unterteilen. Wird eine Zone geheizt oder gekühlt, liegt eine "thermisch konditionierte Zone" vor. Im Falle der Abwesenheit jeglicher Anlagentechnik, wird die Zone als "nicht konditioniert" bezeichnet. \\
Den dritten Schritt der Zonierung stellt die Zusammenfassung von kleinen Gebäudeflächen dar. Bei einem prozentualen Anteil von weniger als 5\% der Gebäudefläche, können kleine Zonen unabhängig von ihrer Nutzung einer Zone zugeteilt werden, die die gleiche Konditionierung aufweist. Bei einem prozentualen Anteil von weniger als 1\% spielt auch die Konditionierung der Zone keine Rolle mehr und die Zone kann einer anderen zugeordnet werden. Ausgenommen von dieser Regelung sind Zonen mit erheblichen inneren Lasten oder Luftwechselzahlen.\\
\\
Im Anschluss an die Zonierung und unabhängig von dieser erfolgt die Einteilung in unterschiedliche Versorgungsbereiche. Ein Versorgungsbereich entspricht dem Bereich eines Gebäudes, welcher durch die gleiche Anlagentechnik abgedeckt wird. Auf Grundlage der Einteilung in Versorgungsbereiche können Teilenergiekennwerte einzelner technischer Anlagen bestimmt werden. Deckt ein Versorgungsbereich mehrere Zonen ab, werden die Energiekennwerte auf die einzelnen Zonen aufgeteilt. \\
\\
Die Bestimmung des Nutzenergiebedarfs erfolgt für jede Zone einzeln. Wärmesenken und Wärmequellen werden anschließend für jede Zone einzeln gegenübergestellt. Die Gegenüberstellung dieser Daten für das gesamte Gebäude ist wenig aussagekräftig, da durch den räumlichen Aspekt Wärmesenken und Wärmequellen eine gegenseitige Beeinflussung nicht zwangsläufig sinnvoll ist.\\
\\
In Teil 10 werden 43 Profile mit Nutzungsrandbedingungen von Nichtwohngebäuden vorgestellt. Diese sind als Richtwerte anzunehmen, wenn keine nach den Regeln der Technik ermittelten Werte über das Gebäude vorhanden sind. Als Nutzungsrandbedingungen gelten Angaben zu Nutzungs- und Betriebszeiten, Beleuchtung, Raumklima, Wärmequellen, Trinkwasser und Gebäudeautomation. In den Profilen sind diese Bereiche jeweils mit Richtwerten abgedeckt. 
Bei der Zusammenlegung von Zonen, die durch unterschiedliche Nutzungen geprägt sind, werden die Richtwerte gemittelt. Ausnahmen bilden in diesem Fall die Nutzungs- und Betriebszeit in Stunden pro Tag sowie in Tagen pro Jahr. Im Bezug auf Heizung und Kühlung werden jeweils die Extreme als Richtwert verwendet,im Fall der Heizung die Maximaltemperatur, im Fall der Kühlung die Minimaltemperatur.  \\
Da das Klima Einfluss auf die Anlagentechnik nimmt, werden im Teil 10 Angaben zu Klimadaten gemacht, die sich aus Strahlungsintensität, Außenlufttemperatur und Windgeschwindigkeit zusammensetzen. Das Klima in Potsdam wird als Referenzklima für Deutschland festgelegt.\\
\\

\chapter{Stand der Technik}
\label{cha:Stand der Technik}

Die Entwicklung der Energieeffizenz wird im Bauwesen durch verschiedene Forschungsprogramme gefördert. Trotz der EnEV und den weiteren gesetzlichen Maßnahmen liegt die Sanierungsrate leidglich jährlich bei weniger als 1\%. Um die energiepolitischen Ziele der Bundesregierung, einen klimaneutralen Gebäudebestand bis 2050, zu erreichen, ist eine Verdopplung dieser Rate erforderlich.  [Energiekonzept,2010] \\
\\
Im Bereich der Wohngebäude hat die Forschung in den letzten Jahren weitere Fortschritte gemacht. Während vor 20 Jahren Nierdrigenergiehäuser der Forschungsstand waren, ist heutzutage der Bau von Häusern möglich, die einen negativen Jahres-Primärenergiebedarf sowie einen negativen Jahres-Endenergiebedarf aufweisen. Dieses Effizienzhaus Plus-Niveau basiert auf drei Säulen: Zum einen muss die Energieeffizienz des Gebäudes optimiert und gleichzeitig der Bedarf an Energie der Haushaltsprozesse minimiert werden.  Der restliche Energiebedarf wird mithilfe erneuerbarer Energien gedeckt. \\
Der Energieverlust wird durch einen Gebäudeentwurf minimiert, der durch einen kompakten Gebäudekörper sowie eine optimale Ausrichtung des Gebäudes charakterisiert ist. Hierbei spielt auch die thermische Zonierung eine Rolle. Weitere Faktoren bilden ein effizienter Wärmeschutz bei Fenstern und der restlichen Umfassungsfläche sowie ein luftdichtes Gebäude. Unabhängig vom Aufbau des Gebäudes ist das Nutzerverhalten sowie die Ausstattung der Haushaltsgeräte. \\
\\
Erneuerbare Energien können sowohl aktiv als auch passiv Einfluss auf die Energiebilanz eines Gebäudes nehmen. Eine passive Einflussnahme liegt beispielsweise vor, wenn aufgrund von Tageslicht keine zusätzliche Beleuchtung eingeschaltet werden muss. Ein weiteres Beispiel beschreibt die durch Sonneneinstrahlung erzeugte Wärmeenergie, die den Heizbedarf eines Gebäudes senken kann. Erneuerbare Energien werden im Fall von Umweltwärme, Geothermie und Biomasse aktiv miteinbezogen. Der Betrieb von Photovoltaik und Windenergie ermöglicht die Produktion von Strom, welcher anschließend in Speicher oder in das Netz von Energieanbietern gelangt. [Wege zum Effizienzhaus Plus, 2016]\\
\\
Um die Errichtung von Nichtwohngebäuden mit Effizienzhaus Plus – Standard zu fördern, ist im Januar 2015 die Richtlinie „Vergabe von Zuwendungen für Modellprojekte hier Förderzweig: Bildungsbauten im Effizienzhaus Plus-Standard“ eingeführt worden. Die Förderung fokussiert das Ziel, Grundlagen für die Markteinführung des Effizienzhaus Plus zu schaffen sowie Marktdurchdringung von vorhandener und hocheffizienter Technologien voranzutreiben. \\
\\
Das Forschungsprojekt Datenbasis Gebäudebestand aus dem Jahr 2010 dient der Ermittlung des Modernisierungstrends im Wohngebäudebestand. Betrachtet wurde lediglich die Sanierung des Wärmeschutzes. Der Studie zufolge beträgt die Modernisierungsrate bei Gebäuden, die vor der ersten Wärmeschutzverordnung erbaut wurden (bis 1978) 1,1\%/a. Der Modernisierungsfortschritt bezogen auf den Wärmeschutz liegt laut Endbericht zwischen 25 und 30\%. Unter Einbeziehung der Gebäude, die nach 1978 erbaut wurden, schrumpft die Modernisierungsrate auf einen Wert von 0,8\%/a. Um eine klimaneutralen Gebäudebestand zu erreichen, muss die Modernisierungsrate wesentlich erhöht werden. Das Ziel des Energiekonzepts der Bundesregierung liegt bei einer Modernisierungsrate von 2\%/a bezogen auf Wohn- sowie Nichtwohngebäude.
Im Bereich der Nichtwohngebäude ist der Gebäudebestand nur unzureichend aufgenommen und die dort vorhandenen Einsparungspotentiale sind somit nicht bekannt. \\
\\

\chapter{Gebäudebestand}
\label{cha:Gebäudebestand}

Wird der Bestand der Nichtwohngebäude betrachtet, zeigt sich, dass dieser nur unzureichend aufgezeichnet ist. 
Nach Betrachtung des Primärenergieverbrauchs bezüglich der jeweiligen Anwendung der Anlagentechnik (Beleuchtung, Kühlung, Heizung und Warmwasser) in Wohngebäuden und Nichtwohngebäuden, ergibt sich Folgendes Bild. Nichtwohngebäude verwenden 41\% des Primärenergieverbrauchs, während Wohngebäude die restlichen 59\% für die Energieversorgung in Anspruch nehmen. Laut Gebäudereport (dena) liegt der Anteil der Raumwärme im Nichtwohngebäudebereich bei 57\%, gefolgt von 35\% für Beleuchtung und jeweils 4\% für Kühlung und Warmwasser.\\
\\
Der Energieverbrauch im Gebäudebereich setzt sich laut Gebäudereport der dena im Jahr 2010 wie folgt zusammen: 35\% der Energie wird von Nichtwohngebäuden und 65\% in Wohngebäuden genutzt. Mit einer Anzahl von 1,8 Millionen Nichtwohngebäuden und 18,2 Millionen Wohngebäuden, folgt, dass Nichtwohngebäude einen hohen Verbrauch aufweisen müssen. [dena, 2012]\\
Die unterschiedlichen anlagentechnischen Bereiche unterteilen sich bei Nichtwohngebäuden wie folgt: Mit einem prozentualen Wert von 71\% deckt die Raumwärme den größten Anteil der Nichtwohngebäude ab. Mit 20\% folgt die Beleuchtung als zweitgrößter Endenergieverbraucher. Warmwasser mit 7\% und Klimakälte 2\% vervollstädigen den Endenergieverbrauch bei Nichtwohngebäuden. Die Datengrundlage ist in diesem Fall das Jahr 2010.\\